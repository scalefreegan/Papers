%%%%%%%%%%%%%%%%%%%%%%%%%%%%%%%%%%%%%%%%%
% Journal Article
% LaTeX Template
% Version 1.3 (9/9/13)
%
% This template has been downloaded from:
% http://www.LaTeXTemplates.com
%
% Original author:
% Frits Wenneker (http://www.howtotex.com)
%
% License:
% CC BY-NC-SA 3.0 (http://creativecommons.org/licenses/by-nc-sa/3.0/)
%
%%%%%%%%%%%%%%%%%%%%%%%%%%%%%%%%%%%%%%%%%
%----------------------------------------------------------------------------------------
%       PACKAGES AND OTHER DOCUMENT CONFIGURATIONS
%----------------------------------------------------------------------------------------
\documentclass[paper=letter, fontsize=12pt]{article}
\usepackage[english]{babel} % English language/hyphenation
\usepackage{amsmath,amsfonts,amsthm} % Math packages
\usepackage[utf8]{inputenc}
\usepackage{float}
\usepackage{lipsum} % Package to generate dummy text throughout this template
\usepackage{blindtext}
\usepackage{graphicx} 
\usepackage{caption}
\usepackage{subcaption}
\usepackage[sc]{mathpazo} % Use the Palatino font
\usepackage[T1]{fontenc} % Use 8-bit encoding that has 256 glyphs
\linespread{2} % Line spacing - Palatino needs more space between lines
\usepackage{microtype} % Slightly tweak font spacing for aesthetics
\usepackage[hmarginratio=1:1,top=32mm,columnsep=20pt]{geometry} % Document margins
\usepackage{multicol} % Used for the two-column layout of the document
%\usepackage[hang, small,labelfont=bf,up,textfont=it,up]{caption} % Custom captions under/above floats in tables or figures
\usepackage{booktabs} % Horizontal rules in tables
\usepackage{float} % Required for tables and figures in the multi-column environment - they need to be placed in specific locations with the [H] (e.g. \begin{table}[H])
\usepackage{hyperref} % For hyperlinks in the PDF
\usepackage{lettrine} % The lettrine is the first enlarged letter at the beginning of the text
\usepackage{paralist} % Used for the compactitem environment which makes bullet points with less space between them
\usepackage{abstract} % Allows abstract customization
\renewcommand{\abstractnamefont}{\normalfont\bfseries} % Set the "Abstract" text to bold
\renewcommand{\abstracttextfont}{\normalfont\small\itshape} % Set the abstract itself to small italic text
\usepackage{titlesec} % Allows customization of titles

\renewcommand\thesection{\Roman{section}} % Roman numerals for the sections
\renewcommand\thesubsection{\Roman{subsection}} % Roman numerals for subsections

\titleformat{\section}[block]{\large\scshape\centering}{\thesection.}{1em}{} % Change the look of the section titles
\titleformat{\subsection}[block]{\large}{\thesubsection.}{1em}{} % Change the look of the section titles
\newcommand{\horrule}[1]{\rule{\linewidth}{#1}} % Create horizontal rule command with 1 argument of height
\usepackage{fancyhdr} % Headers and footers
\pagestyle{fancy} % All pages have headers and footers
\fancyhead{} % Blank out the default header
\fancyfoot{} % Blank out the default footer

\fancyhead[C]{University of Washington $\bullet$ 19 May 2015 $\bullet$ Molecular and Cellular Biology (MCB) Program} % Custom header text

\fancyfoot[C]{\thepage} % Custom footer text
%----------------------------------------------------------------------------------------
%       TITLE SECTION
%----------------------------------------------------------------------------------------
\title{\vspace{-15mm}\fontsize{20pt}{10pt}\selectfont\textbf{Interpreting the complexity of microbes using genomic data}} % Article title
\author{
\large
{\textsc{ Aaron N. Brooks, PhD }}\\[0.1mm]
\normalsize \href{mailto:aaron.brooks@embl.de}{aaron.brooks@embl.de}\\[0.1mm] % Your email address
}
\date{}

%----------------------------------------------------------------------------------------
\begin{document}
\maketitle % Insert title
\thispagestyle{fancy} % All pages have headers and footers

Our planet is teeming with microbes. A single gram of soil (less than a teaspoon) often contains hundreds of millions single-celled organisms representing tens of thousands of different species -- and that is just soil. Microbes are everywhere. They colonize our skin, flourish in our guts, thrive in the oceans, and even grow on our nuclear waste. Collectively, their biomass likely exceeds that of both plants and animals. It should be no surprise, then, that microbes have a profound impact on our planet. Individually and collectively these little creatures sculpt ecosystems and influence our health in fundamental ways, many of which we have yet to understand fully. 

The purpose of my dissertation research was to build computer-based models of microbes learned from the so-called high-throughput "-omics" data that are now relatively easy to collect in the lab and widely available as public resources.  we focused on a very specific. In the end, I successfully constructed models for two species spanning 2 of the 3 kingodms of life (bacteria and archaea).  The models suggest that microbial genomes are controlled in ways that are far more complex and sophisticated than people had previosly imagined. More important, however, I helped establish. We have now 

In popular culture and many scientific circles single-celled microrganisms are considered to be relatively 
\textit{"simple"}. This stereotype makes sense: their genomes are much smaller than ours; they encode far fewer genes; and, clearly, they consist of a single-cell.  Yet, many decades of thoughtful study, we know very little about how they function apart from detailed characterization of in a few organisms, like the much-loved bakers yeast \textit{Saccharomyces cerevisiae} and often-maligned \textit{Escherichia coli}, in a handful of laboratory conditions. What we do know is that these organisms environment-specific response. What is not
entirely clear is how these abilities are encoded in their genomes.\\[1mm]

\noindent\textbf{Purpose:}


%----------------------------------------------------------------------------------------

\end{document}
